
       
\section{Programação Dinâmica}

%%%%%%%%%%%%%%%%%%%%%%%%%%
\begin{frame}
\frametitle{Programação Dinâmica}
\begin{minipage}{0.47\textwidth}
    \begin{itemize}
        \item O que é a PD?
        \item Características
        \item Importância
        \item Exemplo
        
    \end{itemize}
\end{minipage}
\begin{minipage}{0.5\textwidth}
\begin{figure}[ht!]
\begin{center}
\includegraphics[width=1.2\textwidth, height=0.40\textheight]{figures/logo_picat_alex.jpg}
\end{center}
\end{figure}
\end{minipage}
\end{frame}
%%%%%%%%%%%%%%%%%%%%%%%%%%


\begin{frame}[fragile]
%[fragile, allowframebreaks=0.9]

    \frametitle{Programação Dinâmica (PD) -- I}

   \begin{block}{}
     \begin{itemize}
      \item Uma poderosa \textcolor{magenta}{\textit{técnica de programação}}  que  contorna a complexidade de certos problemas
      exponenciais
      
       \pause
       \item O problema \textbf{deve} apresentar uma \textcolor{magenta}{\textit{\underline{regra de recorrência}}}
       
      \pause
      \item A idéia é que todos os cálculos feitos a partir desta \textit{regra de recorrência},
     sejam armazenados numa \textit{tabela dinâmica} e consultados para reuso de novos cálculos de outras
      instâncias
      
      \pause
      \item Esta \textcolor{magenta}{\textit{técnica de programação}} 
      utiliza uma \textit{tabela dinâmica} nos cálculos intermediários,
      evitando a repetição do que já foi calculado anteriormente, é conhecida como:
      \textcolor{magenta}{Programação Dinâmica}, ou simplesmente:
       \underline{\textcolor{magenta}{\textbf{PD}}}

    \end{itemize}
    
    \end{block}
    
\end{frame}



\begin{frame}[fragile]
\frametitle{Programação Dinâmica (PD) -- II}

\begin{figure}[!htb]
\centering
\includegraphics[width=0.95\textwidth, height=0.70\textheight]{figures/ilustra_PD.png}
\caption{Conceitos da Programação Dinâmica -- (PD) -- Resumos}
\end{figure}
\end{frame}






\begin{frame}[fragile]
%[fragile, allowframebreaks=0.9]

    \frametitle{Programação Dinâmica (PD) -- III}

   \begin{block}{}
     \begin{itemize}

      \item Como Picat usa a recursão, na programação em lógica, nada mais
      natural do que esta ter a PD disponível 

       \pause
       \item O comando que cria uma tabela para um determinado predicado é o  \textcolor{magenta}{\textbf{\textit{tabling}}}
 
        \pause
       \item O \textit{tabling}  é um dos elementos fortes do planejador do Picat (módulo \textit{planner})

        \pause
       \item Assim a PD, faz a complexidade ser espacial devido o uso de memória em seus cálculos intermediários

        \pause
       \item O exemplo escolhido para ilustrar a PD em Picat, veio do texto \textit{Modeling and Solving AI
        Problems in Picat}, de Roman Barták e Neng-Fa
    \end{itemize}
    
    \end{block}
    
\end{frame}



\begin{frame}[fragile]
%[fragile, allowframebreaks=0.9]

\frametitle{Exemplo de Uso da Programação Dinâmica -- (PD)}

\begin{itemize}
  \item Seja o binômio ${\left(x + y\right)}^n$, conhecido como \textit{Binômio de Newton}

  \pause 
  \item Casos particulares são:
  \item  ${\left(x + y\right)}^0 = 1$
  \item  ${\left(x + y\right)}^1 = x + y$
  \item  ${\left(x + y\right)}^2 = x^2 + 2xy + y^2$
  
  \pause
  \item  ${\left(x + y\right)}^2 = x^2y^0 + 2x^1y^1 + x^0y^2$
  \item  ${\left(x + y\right)}^3 = x^3y^0 + 3x^2y^1 + 3x^1y^2 + x^0y^3$
  \item  ${\left(x + y\right)}^4 = x^4y^0 + 4x^3y^1 + 6x^2y^2 + 4x^1y^3 + x^0y^4.$
  \item  .......
  \pause
 \item Como obter estes coeficientes  polinômios?  

\end{itemize}
    
\end{frame}


\begin{frame}[fragile]
%[fragile, allowframebreaks=0.9]

\frametitle{Exemplo de Uso da Programação Dinâmica -- (PD)}

\begin{figure}[!htb]
\centering
\includegraphics[width=0.70\textwidth, height=0.60\textheight]{figures/pascal_triangle_01.jpg}
%%%prolog/scale=0.47
%\label{fig_nos_estados}
\caption{O triângulo de Pascal}
\end{figure}
\end{frame}


%%%%%%%%%%%%%%%%%%%%%%%%%%%%%%%%%%%%%%%
\begin{frame}[fragile]
%[fragile, allowframebreaks=0.9]

\frametitle{Exemplo de Uso da Programação Dinâmica -- (PD)}

\begin{figure}[!htb]
\centering
\includegraphics[width=0.850\textwidth, height=0.650\textheight]{figures/pascal_triangle_02.jpg}
%%%prolog/scale=0.47
%\label{fig_nos_estados}
\caption{O triângulo de Pascal -- Coeficientes Binomiais}
\end{figure}
    
\end{frame}

%%%%%%%%%%%%%%%%%%%%%%%%%%%%%%%%%%%%%%%

\begin{frame}[fragile]
%[fragile, allowframebreaks=0.9]

\frametitle{Formulação Matemática -- I}

\begin{itemize}
  \item O \textit{coeficiente binomial}, também chamado de \textit{número binomial}, 
de um número $n$, na classe $k$, consiste no número de combinações de $n$ termos, $k$ a $k$. 

\pause
  \item O número binomial de um número $n$, na classe $k$, pode ser escrito como:

$$ {n \choose k}= \frac {n!}{k!(n-k)!}=\frac {n(n-1)(n-2)\cdots(n-k+1)}{k!}$$
\end{itemize}   
    
\end{frame}



\begin{frame}[fragile]
%[fragile, allowframebreaks=0.9]

\frametitle{Formulação Matemática -- II}

\begin{itemize}
  \item Alternativa ao cálculo do fatorial, tem-se a relação de Stiffel:

 $$ {n\choose k}={n-1\choose k-1}+{n-1\choose k}$$  
    

\pause
  \item  O coeficiente binomial é muito utilizado no Triângulo de Pascal, onde o 
  termo na linha $n$ e coluna $k$ é  dado por: ${n-1 \choose k-1}$
  
  \pause
  \item A fórmula de Stiffel é \textcolor{red}{recorrente} e diretamente escrita em Picat.\\
  Veja os códigos ...
\end{itemize}   
    
\end{frame}


\begin{frame}[fragile] 

\frametitle{Código em Partes}

\begin{footnotesize}
\begin{verbatim}
import datetime.   %%% para o statistics
import util.


table
c(_, 0) = 1.
c(N, N) = 1.
c(N,K) = c(N-1, K-1) + c(N-1, K).
\end{verbatim}
\end{footnotesize}    

\begin{center}
\textcolor{magenta}{Esta fórmula é semelhante com a sequência de Fibonacci, vista
na seção de recursividade, mas aqui temos 2 argumentos em \texttt{c(N,K)}. 
Logo, o número de 
variações de cada coeficiente \texttt{c(N,K)}, 
cresce muito os cálculos repetidos!}
\end{center}

\end{frame}



\begin{frame}[fragile] 
\frametitle{Código em Partes}

\begin{footnotesize}
\begin{verbatim}
main  ?=>  
    statistics(runtime,_), % faz uma marca do 1o. statistics
    N = 10, %% ateh uns 30 ... são números grandes ... fatorial
     foreach(I in 0  .. N)
        foreach(J in 0  ..  I)
             printf("  %d", c(I,J))
           end,
          printf(" \n"),
      end, 
    statistics(runtime, [T_Picat_ON, T_final]),
    T = (T_final) / 1000.0, %%% está em milisegundos
    printf("\n CPU time %f em SEGUNDOS ", T),
    printf("\n OVERALL PICAT CPU time %f em SEGUNDOS ", T_Picat_ON/1000.0),
    printf(" \n =========================================\n ")
    %%% , fail descomente para multiplas solucoes
    .
main => printf("\n Para uma solução .... !!!!" ) .
\end{verbatim}
\end{footnotesize}
    
\end{frame}


\begin{frame}[fragile]
 \frametitle{Código Completo}

\begin{itemize}
  \item Acompanhar as explicações do código de:\\
\url{https://github.com/claudiosa/CCS/blob/master/picat/coeficiente_binomial_PD.pi}

  \item Confira a execuç\~ao
\end{itemize}
\end{frame}
%%%%%%%%%%%%%%%%%%%%%%%%%%%%%%%%%%%%%%%%%%%%%%%%%%%%%%%%%%%%%%%
\begin{frame}[fragile]
%[fragile, allowframebreaks=0.9]

\frametitle{Saída}
\begin{footnotesize}
\begin{verbatim}
[ccs@gerzat picat]$ picat coeficiente_binomial_PD.pi 
  1 
  1  1 
  1  2  1 
  1  3  3  1 
  1  4  6  4  1 
  1  5  10  10  5  1 
  1  6  15  20  15  6  1 
  1  7  21  35  35  21  7  1 
  1  8  28  56  70  56  28  8  1 
  1  9  36  84  126  126  84  36  9  1 
  1  10  45  120  210  252  210  120  45  10  1 

 CPU time 0.000000 em SEGUNDOS 
 OVERALL PICAT CPU time 0.009000 em SEGUNDOS  
 =========================================
\end{verbatim}

\end{footnotesize}    

\end{frame}

%%%%%%%%%%%%%%%%%%%%%%%%%%%%%%%%%%%%%%%%%%%%%%%%%%%%%%%%%%%%%%%
\begin{frame}[fragile]
\frametitle{Reflexões sobre PD}


\begin{itemize}
  \item Há outros métodos para se resolver estes problemas

  \pause
  \item O comando \textit{tabling} é a base do módulo \textit{planner},
  usado para resolver \underline{problemas de planejamento}

  \pause
  \item A PD é uma estratégia de programação bem poderosa
  
  \pause
  \item Assunto das próximas seções:  \underline{Planejamento} e PR

  
\end{itemize}

\end{frame}

%%%%%%%%%%%%%%%%%%%%%%%%%%%%%%%%%%%%%%%%%%%%%%%%%%%%%%%%%%%%%%%
