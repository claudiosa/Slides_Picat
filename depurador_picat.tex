\section{Depuração}
%%%%%%%%%%%%%%%%%%%%%%%%%%
\begin{frame}[fragile]
\frametitle{Depuração}
\begin{minipage}{0.47\textwidth}
    \begin{itemize}
        \item O que é depuração?
        \item Como funciona no Picat? 
        \item Exemplo de uso
    \end{itemize}
\end{minipage}
\begin{minipage}{0.5\textwidth}
\begin{figure}[ht!]
\begin{center}
\includegraphics[width=1.2\textwidth, height=0.40\textheight]{figures/logo_picat_alex.jpg}
\end{center}
\end{figure}
\end{minipage}
\end{frame}
%%%%%%%%%%%%%%%%%%%%%%%%%%
\begin{frame}[fragile]

    \frametitle{Depuração -- Quando usar?}

    \begin{itemize}
      \item Depuração,  \textit{trace} ou \textit{debug} -- o que é?
      \pause
      Mostrar estado de variáveis, funções e predicados durante a execução de um código
    
      \item Depuração ou \textit{trace} -- quando usar?
      \pause
     \textbf{\textcolor{violet}{Aprender detalhes da linguagem}}

  \pause
  \item Depuração ou \textit{trace} $\Rightarrow $      \textbf{\textcolor{violet}{para descobrir o erro!}}

\pause
    \item Em Picat não há um ambiente gráfico de depuração, 
    ao contrário do SWI-Prolog. 
      \pause
    Eis uma oportunidade de um projeto aqui....
        
    \item Contudo, o \textit{trace} do Picat faz tudo ...
      \pause
    
    \end{itemize}
\end{frame}

\begin{frame}[fragile]

\frametitle{Depuração -- Como funciona um predicado?}

\begin{figure}[!htb]
\begin{center}
\includegraphics[width=0.90\textwidth, height=0.5\textheight]{figures/trace01.pdf}
%\caption{Realizar buscas com regiões reduzidas -- promissoras (regiões factíveis de soluções)}
\end{center}
\end{figure}

\end{frame}


\begin{frame}[fragile]


\frametitle{Depuração -- Internamente  tem o \textit{backtracking}!}
\begin{figure}[!htb]
\begin{center}
\includegraphics[width=0.90\textwidth, height=0.6\textheight]{figures/trace02.pdf}
\caption{Levemente diferente do que é encontrado na bibliografia!}
\end{center}
\end{figure}

\end{frame}


\begin{frame}[fragile]

\frametitle{Depuração -- uma regra composta}

\begin{figure}[!htb]
\begin{center}
\includegraphics[width=0.90\textwidth, height=0.6\textheight]{figures/trace03.pdf}
%\caption{Realizar buscas com regiões reduzidas -- promissoras (regiões factíveis de soluções)}
\end{center}
\end{figure}

\end{frame}


%%%%%%%%%%%%%%%%%%%%%%%%%%%%%%%%%%%%%%%%%%%%%%%%%%%%
\begin{frame}[fragile]

    \frametitle{Depuração}

    \begin{itemize}
    
      \item Picat tem \textbf{3 modos de execução}: \textit{trace},
      \textit{non-trace} e \textit{spy}
      \pause
      \item  \textit{non-trace}: é a execução normal ou \textit{default} -- vista até o momento

      \pause
      \item  \textit{trace} ou \textit{debug}: vê passo-a-passo de cada variável, predicado ou      função,  do código inteiro
%      \pause
%      \item Uso para fins diversos
      \pause
      \item \textit{spy}: \textit{salta} para um predicado ou função específica,
       se iniciando a depuração a partir deste ponto

      \pause
      \item Devem ser habilitados e desabilitados no \underline{\textbf{modo interpretado}}
    
      \item \textbf{\textcolor{red}{Há funções e predicados que não tem \textit{backtracking}.}}\\
      \pause
       Exemplos: \texttt{nl}, \texttt{printf}, \texttt{read\_...}, etc
       
    \end{itemize}
\end{frame}
%%%%%%%%%%%%%%%%%%%%%%%%%%%%%%%%%%%%%%%%%%%%%%%%%%%%


\begin{frame}[fragile]

\frametitle{Uso básico do \textit{trace}}

\begin{enumerate}
  \item Entre no interpretador do Picat
  
    \item Ative o modo \textit{trace} (ou \textit{debug}), com o comando: \textit{trace}
    
    \item Compile e carregue o programa desejado 
    a depurar, com o comando: \texttt{cl(`....').}
    
    \item Execute a parte desejada do código
    
    \item Vá dando \texttt{Enter} no modo \textit{passo-a-passo}
    
    \item Digite `\textbf{h}' para o \texttt{help}\texttt{\texttt{}}  e ver opções disponíveis
    
    \item Digite `\textbf{a}' para o \texttt{abortar}\texttt{} o modo trace
      
\end{enumerate}
\begin{center}
\textbf{\textcolor{magenta}{Em qualquer modificação do código fonte no editor, repita os passos acima, desde o início!}}
\end{center}

\end{frame}

%\subsection{Exemplo de uso do \textit{trace}}
\begin{frame}[fragile]

\frametitle{Exemplo:  \texttt{X + Y = 22}}

\begin{footnotesize}
\begin{verbatim}
main ?=> 
    %%%%% ACHE X e Y tal que  X + Y = 22 
    regra(X,Y, 22) , 
    printf("SAIDAS ==> X: %w Y: %w \n", X, Y)  ,  
    false.
main =>  printf("\n Não há mais soluções! \n"). 
%%%%%%%%%%%%%%%%%%%%%%%%%%%%%%%%%%%%%%%%%%%%%% 
regra(X,Y, R) =>   
      p(X),
      q(Y),
      R = X + Y.
%%%%%%%%%%%%%%%%%%%%%%%%%%%%%%%%%%%%%%%%%%%%%
index(-)      
     p(2).     p(1).     p(0).
index(-)      
     q(20).    q(22).
%%%%%%%%%%%%%%%%%%%%%%%%%%%%%%%%%%%%%%%%%%%%%%   
\end{verbatim}
\end{footnotesize}
\end{frame}

%%%%%%%%%%%%%%%%%%%%%%%%%%

%\subsection{Exemplo de uso do \textit{trace}}
\begin{frame}[fragile]

\frametitle{Saída -- \textit{trace}}

\begin{scriptsize}
\begin{verbatim}
Picat> trace
Note: you need to recompile programs in debug mode for tracing
yes
{Trace mode}
Picat>  cl('trace_exemplo_01')
Compiling:: trace_exemplo_01.pi
trace_exemplo_01.pi compiled in 3 milliseconds
loading...
yes
{Trace mode}
Picat> q(X), writeln(x = X), false.
   Call: (1) q(_10b38) ?
?  Exit: (1) q(20) ?
   Call: (2) writeln(x = 20) ?
x = 20
   Exit: (2) writeln(x = 20) ?
   Redo: (1) q(20) ?            >>>> DEVIDO o false
   Exit: (1) q(22) ?
   Call: (3) writeln(x = 22) ?
x = 22
   Exit: (3) writeln(x = 22) ?
no
{Trace mode}
\end{verbatim}
\end{scriptsize}
\end{frame}


\begin{frame}[fragile]

\frametitle{Saída -- \textit{spy}}

\begin{scriptsize}
\begin{verbatim}
Picat> cl('trace_exemplo_01')
>>>>>>>>>>>>>>>>>>>>>>> OMITIDO LINHAS AQUI
{Trace mode}
Picat> spy =    
Spy point set on =.
>>>>>>>>>>>>>>>>>>>>>>> OMITIDO LINHAS AQUI
{Spy mode}
Picat> main
   Call: (5) 22 = 22 ?
   Exit: (5) 22 = 22 ?
?  Exit: (2) regra(2,20,22) ?
   Call: (6) printf(['\n',' ','X',:,' ','%',w,' ','Y',......],2,20) ?l

 X: 2 Y: 20    Call: (7) 22 = 24 ?
   Fail: (7) 22 = 24 ?
?  Exit: (3) p(1) ?l
   Call: (9) 22 = 21 ?l
>>>>>>>>>>>>>>>>>>>>>>> OMITIDO LINHAS AQUI  
   Exit: (13) 22 = 22 ?l
 X: 0 Y: 22 
 Não há mais soluções! 
>>>>>>>>>>>>>>>>>>>>>>> OMITIDO LINHAS AQUI
{Spy mode}
\end{verbatim}
\end{scriptsize}
\end{frame}




\begin{frame}[fragile]

\frametitle{Considerações de Uso: \textit{trace} e \textit{spy} }

\begin{itemize}
  \item Estes comandos são úteis e se encontram em modo \textit{beta} de uso
  
  \item Mas, tem \textbf{TUDO} que se precisa para descobrir um erro
    \pause
    \item   \textcolor{magenta}{\textbf{Os números após a chamadas, representam a sequência na pilha de cada predicado!}} 

  
  \pause
  \item A idéia de se programar com Picat e outras linguagens
  declarativas é: 

  \pause
  \textcolor{magenta}{\textbf{a cada linha de código escrita segue-se por um teste isolado!}}

  \pause
  \item Ou seja,  use o \textit{trace} e o \textit{spy} em partes do código!

   \pause
    \item   \textcolor{magenta}{\textbf{Nos modos \textit{trace} e \textit{spy}, veja as opções do \textit{help} ou (h).\\
    Experimente-as sem receios. Em caso de dúvida: a (\textit{abort})!}} 

  \pause
  \item Ao se ganhar confiança com a linguagem, voce vai omitindo alguns pontos!

  
\end{itemize}

\end{frame}


\begin{frame}[fragile]
\frametitle{Reflexões}

\begin{itemize}
  \item \textcolor{magenta}{Se aprende os detalhes da linguagem com um \textit{trace}}
  
  \pause
    \item Não se assuste com modo \texttt{trace} na console
  
  
  \pause
    \item Atenção na sequência de uso: 
    
    \begin{enumerate}
      \item Ative o modo \textcolor{magenta}{\textit{trace}}
      \item Compile e carregue o seu fonte no modo trace, com o comando \texttt{cl(`.....')}.
      \item Execute o que desejares do código ou ative alguns \textcolor{magenta}{\textit{spy}}
      \item Para desativar  \textcolor{magenta}{\textit{notrace}}
    \end{enumerate}
    \pause
    Em caso de mudança no código, repita os passos acima
       
  \pause
    \item Outra alternativa ao \textcolor{magenta}{\textit{trace}} $\Rightarrow$ modo \textcolor{magenta}{\textit{spy}}, para desativar
    \textcolor{magenta}{\textit{nospy}}
    
    
\pause
    \item Finalmente, \textit{boas depurações}!        
  
\end{itemize}


\end{frame}
