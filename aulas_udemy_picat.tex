%%%%%%%%%%%%%%%%%%%%%%%%%%%%%%%%%%%%%%%%%%%%%%%%%%%%%%%%%%%%%%
\section{Apresentação ao Curso de PICAT}

%%%The \pause command internally uses \onslide (see §9.1 of the beamer manual), so it does employ overlay specifications.


\begin{frame}[fragile]

  \frametitle{Apresentação ao Curso de PICAT -- I}
  \begin{itemize}
    \item O que é o PICAT?
    \pause
       \begin{itemize}
			\item Uma linguagem de programação de propósitos gerais
			\item Uma evolução do PROLOG (consagrada linguagem dos primórdios da IA)
			\item Tem elementos de Python, Prolog e Haskell
		\end{itemize}

    \item Uso e finalidades do PICAT:
    \pause
       \begin{itemize}
			\item Uso de programas gerais, simples a complexos
			\item Provê suporte há vários solvers na área de Pesquisa Operacional
			\item Área: IA, programação por restrições, programação inteira, planejamento,
			combinatória, etc
		\end{itemize}

   \end{itemize}

  \end{frame}
    
%\framebreak
\begin{frame}[fragile]
  \frametitle{Apresentação ao Curso de PICAT -- II}
  \begin{itemize}

    \item Este curso é dirigido a voce?
  \pause
    \item Requisitos:
   \pause
		\begin{itemize}
			\item De conhecimento:
			\item Dedicação
			\item 
		\end{itemize}

    \item Requisitos computacionais:
    \pause
    Um computador qualquer (arquitetura 16, 32 ou 64 bits), com Linux ou Windows,
    que tenha um compilador C instalado completo, preferencialmente.

  \end{itemize}

\end{frame}


    
\begin{frame}[fragile]
  \frametitle{Apresentação ao Curso de PICAT -- III}
  \begin{itemize}

      \item Comunidade e ações: \url{http://picat-lang.org}

   \item Que tópicos serão cobertos no curso?
  \end{itemize}

\end{frame}
