%%%%%%%%%%%%%%%%%%%%%%%%%%%%%%%%%%%%%%%%%%%%%%%%%%%%%%%%%%%%%%
\section{Apresentação ao Curso de PICAT}

%%%The \pause command internally uses \onslide (see §9.1 of the beamer manual), so it does employ overlay specifications.


\begin{frame}[fragile]

  \frametitle{Apresentação ao Curso de PICAT -- I}
  \begin{itemize}
    \item O que é o PICAT?
    \pause
       \begin{itemize}
			\item Uma linguagem de programação de propósitos gerais
			\item Uma evolução do PROLOG (consagrada linguagem dos primórdios da IA)
			\item Tem elementos das linguagens Python, Prolog e Haskell
		\end{itemize}

    \item Uso e finalidades do PICAT:
    \pause
       \begin{itemize}
			\item Uso de programas gerais: de simples à complexos (uma reflexão)
			\item Provê suporte há vários \textit{solvers} na área de Pesquisa Operacional
			\item Área: IA, programação por restrições, programação inteira, planejamento,
			combinatória, etc
		\end{itemize}

   \end{itemize}

  \end{frame}
    
%\framebreak
\begin{frame}[fragile]
  \frametitle{Apresentação ao Curso de PICAT -- II}
  \begin{itemize}

    \item Este curso é dirigido a voce?
  \pause
    \item Requisitos:
   \pause
		\begin{itemize}
			\item Conhecimento: noções de lógica matemática 
			(proposional e primeira-ordem), matemática elementar, 
			e alguma outra linguagem de programação

			\item Dedicação: depende de você
		\end{itemize}
		
  \pause
    \item Motivação:
   \pause
		\begin{itemize}
			\item Dependendo de sua dedicação, ao final voce vai estar apto a resolver problemas
			computacionais de simples à difíceis

			\item Difícil: muitas linhas de código e muito conhecimento de algoritmos seriam
			necessários
			
			\item Com Picat, há sofisticados esquemas prontos para se construir programas.

		\end{itemize}

  \end{itemize}

\end{frame}


    
\begin{frame}[fragile]
  \frametitle{Apresentação ao Curso de PICAT -- III}
  \begin{itemize}

						
    \item Requisitos computacionais:\\
    \pause
    um computador qualquer (arquitetura 16, 32 ou 64 bits), com Linux, Mac ou Windows,
    que tenha um compilador C instalado completo, preferencialmente.

    \item Comunidade e ações: \url{http://picat-lang.org}
    
    \pause
    \item Códigos e este material, sempre atualizados em: 

    \pause
    \begin{itemize}
      \item  Este PDF e seu texto original:  \url{http://github.com/claudiosa/Slides_Picat} (em código \LaTeX)
     \item   Os códigos fontes dos programas:  \url{http://github.com/claudiosa/CCS/picat}
    \end{itemize}

			\item Além do material aqui disponível em PDF, o mais importante  do curso
			 vai estar na interatividade
			da minha \textbf{apresentação oral}. 
			
    
  \end{itemize}

\end{frame}

    
\begin{frame}[fragile]
  \frametitle{Apresentação ao Curso de PICAT -- IV}
  \begin{itemize}

    \item Há alguns pontos do curso que estão repetidos: \textit{propositalmente}!\\
    \pause
    Reforça os erros que cometi um dia!

    \pause
    \item As aulas aqui apresentadas \textbf{não} serão regravadas!
        
    \pause 
    \item Contudo, o texto completo, incluindo os fontes dos programas: \textbf{SIM}\\
    Pois sempre há perguntas, melhoramentos, etc, que elucidam os pontos aqui abordados.
    
    \pause 
    \item Na parte teórica da definição do Picat, mantive os padrões 
    descritos no manual da linguagem (\url{http://picat-lang.org}).
    
  \end{itemize}

\end{frame}





    
\begin{frame}[fragile]
  \frametitle{Apresentação ao Curso de PICAT -- V}
  \begin{itemize}

    \item Além desta  apresentação do curso, voce pode assistir uma
    parte deste curso em aulas que fiz para o Youtube, há alguns anos atrás:

    \pause
    \item Videoaula 01: Introdução ao PICAT\\
    \textbf{\url {https://www.youtube.com/watch?v=0DmTyFFQPK8}}

    \pause 
    \item Videoaula 02: Tipos de Dados do PICAT\\
    \textbf{\url {https://www.youtube.com/watch?v=7fPKPd0ZDnc}} 
    
    \item Estas videoaulas forem refeitas e  encontram-se com uma outra abordagem
    neste curso.
    
  \end{itemize}

\end{frame}




\begin{frame}[fragile]
  \frametitle{Apresentação ao Curso de PICAT -- VI}
  \begin{itemize}

						
    \item Assim, ao final deste curso terás uma sólida visão  de uma ferramenta
    computacional, utilizada em várias áreas tais como: modelagem matemática, IA,
    Pesquisa Operacional, etc

    \pause
    \item Ao final voce vai conseguir resolver problemas com alguma complexidade e ler
    códigos de grandes programadores da área: Barták, Neng-Fa, Hakank, Dymichenko, etc    
    
    \pause
		\item Em resumo, este material é  um guia para o seu desenvolvimento,
		 com explicações nestas aulas, que funcionam como um \textit{atalho} de
		 horas de estudo sobre vários temas apresentados.
   
    \pause
    \item Tópicos   cobertos no curso: ver índice
  \end{itemize}

\end{frame}
