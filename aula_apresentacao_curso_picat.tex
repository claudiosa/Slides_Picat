%%%%%%%%%%%%%%%%%%%%%%%%%%%%%%%%%%%%%%%%%%%%%%%%%%%%%%%%%%%%%%
\section{Apresentação ao Curso de PICAT}

%%%The \pause command internally uses \onslide (see §9.1 of the beamer manual), so it does employ overlay specifications.

%%%%%%%%%%%%%%%%%%%%%%%%%%
\begin{frame}[fragile]
\frametitle{Apresentação ao Curso de PICAT}
\begin{minipage}{0.47\textwidth}
    \begin{itemize}
        \item A linguagem PICAT
        \item Requisitos e recursos
        \item O que esperar do curso?
        \item Agenda do curso
        \item Abrangência

    \end{itemize}
\end{minipage}
\begin{minipage}{0.5\textwidth}
\begin{figure}[ht!]
\begin{center}
\includegraphics[width=1.2\textwidth, height=0.40\textheight]{figures/logo_picat_alex.jpg}
\end{center}
\end{figure}
\end{minipage}

\pause
\begin{center}
\textbf{Em resumo: Uma visão clara e precisa do que é o curso!}
 \end{center}


\end{frame}
%%%%%%%%%%%%%%%%%%%%%%%%%%


\begin{frame}[fragile]

  \frametitle{Apresentação ao Curso de PICAT -- I}
  \begin{itemize}
    \item O que é o PICAT?
    \pause
       \begin{itemize}
			\item Uma linguagem de programação de propósitos gerais
			\item Uma evolução do PROLOG (consagrada linguagem dos primórdios da IA)
			\item Tem elementos das linguagens Python, Prolog e Haskell
		\end{itemize}

    \item Uso e finalidades do PICAT:
    \pause
       \begin{itemize}
			\item Uso de programas gerais: de simples à complexos (uma reflexão)
			\item Provê suporte há vários \textit{solvers} na área de Pesquisa Operacional
			\item Área: IA, programação por restrições, programação inteira, planejamento,
			combinatória, etc
		\end{itemize}

   \end{itemize}

  \end{frame}
    
%\framebreak
\begin{frame}[fragile]
  \frametitle{Apresentação ao Curso de PICAT -- II}
  \begin{itemize}

    \item Este curso é dirigido a você?
  \pause
    \item Requisitos:
   \pause
		\begin{itemize}
			\item Conhecimento: noções de lógica matemática 
			(proposional e primeira-ordem), matemática elementar, 
			e alguma outra linguagem de programação

			\item Dedicação: depende de você
		\end{itemize}
		
  \pause
    \item Motivação:
   \pause
		\begin{itemize}
			\item Dependendo de sua dedicação, ao final você vai estar apto a resolver problemas
			computacionais de simples à difíceis

			\item Difícil: muitas linhas de código e muito conhecimento de algoritmos seriam
			necessários
			
			\item Com Picat, há sofisticados esquemas prontos para se construir programas.

		\end{itemize}

  \end{itemize}

\end{frame}


    
\begin{frame}[fragile]
  \frametitle{Apresentação ao Curso de PICAT -- III}
  \begin{itemize}
						
    \item Recursos computacionais:\\
    \pause 
    Binários disponíveis para Linux, Mac e Windows
     e Código fonte (em C) também disponível

    \item Comunidade e ações: \url{http://picat-lang.org}
    
    \pause
    \item Códigos e este material, sempre atualizados em: 

    \pause
    \begin{itemize}
      \item  O material do curso \textbf{completo} e \textbf{sempre atualizado} em PDF,
      aqui na plataforma, no material para download da $1a.$ aula
      
     \item   Os códigos fontes dos programas:  \url{http://github.com/claudiosa/CCS/picat}
    \end{itemize}

			\item Além do material aqui disponível em PDF, o mais importante  do curso
			 vai estar na interatividade
			da minha \textbf{apresentação oral}. 
			
    
  \end{itemize}

\end{frame}

    
\begin{frame}[fragile]
  \frametitle{Apresentação ao Curso de PICAT -- IV}
  \begin{itemize}

    \item Há alguns pontos do curso que estão repetidos: \textit{propositalmente}!\\
    \pause
    Reforça os erros que cometi um dia!

    \pause
    \item As aulas aqui apresentadas \textbf{não} serão regravadas!
        
    \pause 
    \item Contudo, o texto completo, incluindo os fontes dos programas: \textbf{SIM}\\
    Pois sempre há perguntas, melhoramentos, etc, que elucidam os pontos aqui abordados.
    
    \pause 
    \item Na parte teórica da definição do Picat, mantive os padrões 
    descritos no manual da linguagem (\url{http://picat-lang.org}).
    
  \end{itemize}

\end{frame}

    
\begin{frame}[fragile]
  \frametitle{Apresentação ao Curso de PICAT -- V}
  \begin{itemize}

    \item Além desta  apresentação do curso, você pode assistir uma
    parte deste curso em aulas  que fiz para o YouTube, há alguns anos atrás:

    \pause
    \item Videoaula 01: Introdução ao PICAT\\
    \textbf{\url {https://www.youtube.com/watch?v=0DmTyFFQPK8}}

    \pause 
    \item Videoaula 02: Tipos de Dados do PICAT\\
    \textbf{\url {https://www.youtube.com/watch?v=7fPKPd0ZDnc}} 
    
    \item Estas aulas são introdutórias, o curso vai muito além destes
    assuntos.
    
    \pause 
    \item Estas videoaulas forem refeitas e  encontram-se com uma outra abordagem
    neste curso. 
    
  \end{itemize}

\end{frame}




\begin{frame}[fragile]
  \frametitle{Apresentação ao Curso de PICAT -- VI}
  \begin{itemize}

						
    \item Assim, ao final deste curso terás uma sólida visão  de uma ferramenta
    computacional, utilizada em várias áreas tais como: modelagem matemática, IA,
    Pesquisa Operacional, etc

    \pause
    \item Ao final você vai conseguir resolver problemas com alguma complexidade e ler
    códigos de grandes programadores da área: Barták, Neng-Fa, Hakank, Dymichenko, etc    
    
    \pause
		\item Em resumo, este material é  um guia para o seu desenvolvimento,
		 com explicações nestas aulas, que funcionam como um \textit{atalho} de
		 horas de estudo sobre vários temas apresentados.
   
    \pause
    \item A seguir os temas cobertos no curso com PICAT:
  \end{itemize}

\end{frame}


\subsection{Conteúdos do Curso}
			
\begin{frame}[fragile]
  \frametitle{Conteúdo}
  \begin{enumerate}

					
    \item  \underline{Introdução} ($\approx$ 27 min):\\
    Histórico, paradigmas de linguagens, usando o Picat, etc

    \pause
    \item \underline{Tipos de Dados e Variáveis} ($\approx$ 28 min):\\
Tipos de Dados, Variáveis, Unificação e Atribuição, Tabela de Operadores, Operadores Especiais,
 Exemplos
    
    \pause
		\item \underline{Predicados e Funções} ($\approx$ 32 min):\\
Casamento de padrões, funções, regras, fatos, metas, exemplos
		
    \pause
    \item \underline{Estruturas de Decisão, Laços e Iteradores} ($\approx$ 28 min):\\
Estruturas de decisão, iteradores, funções e predicados especiais, 
entradas e saídas, exemplos
    
    \pause
		\item  \underline{Recursão} ($\approx$ 28 min):\\
     Conceitos de recursão, conceito de \textit{backtracking}, o paradigma
     de pensar e programar recursivamente, exemplos
\end{enumerate}

\end{frame}




			
\begin{frame}[fragile]
  \frametitle{Conteúdo}
  
  \begin{enumerate}

   \setcounter{enumi}{5}
    \item  \underline{Listas} ($\approx$ 36 min):\\
    Definição, como o Picat opera as listas, exemplos


    \pause
    \item  \underline{Buscas} ($\approx$ 40 min):\\
    Definições, uso, abrangência, exemplos

    \pause
    \item \underline{Programação Dinâmica (PD)}:\\
        Definições, uso, abrangência, exemplos

    
    \pause
    \item \underline{Planejamento}:\\
        Definições, o módulo \textit{planner}, uso, abrangência, exemplos

    \pause
		\item  \underline{Programação por Restrições (PR)}:\\
      Definições, o módulo \textit{cp}, uso, abrangência, exemplos (03).
      Técnicas de PR.

    \pause
		\item  \underline{Conclusões}:\\
    Retrospectiva, tendências, o que ficou faltando, dicas de programação, etc
    
\end{enumerate}

\end{frame}



\begin{frame}[fragile]
  \frametitle{Agradecimentos}


						



\end{frame}
